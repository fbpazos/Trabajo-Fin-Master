\chapter{Conclusions}
Overall, we have suggested a tool that is legible and comprehensible for students as well as a UI that enables them to comprehend the various numerical methods that they will use regularly in mathematics and computer science classes as well as in their professional jobs. In their professional jobs, students will typically use pre-implemented libraries, but they will be grown enough to understand why they are there and how they can be improved.

The presented answer is not only straightforward in nature, but has also undergone a series of programmatic and quality tests, as well as critical discussions about potential implementations, so we can ensure a degree of quality in what has been presented.

This endeavor, like any other, has space for further study and development. A viable extension of this work could, for example, delve deeply into the empirical substantiation of the tool by conducting a usage study with a group of students to assess the tool's effectiveness in aiding them in comprehending the numerical methods they are using. Additionally, a comparison study of the proposed tool with other current tools on the market to assess its strengths and weaknesses.


The suggested addition of new modules for numerical methods such as finite-difference methods and graph algorithms would be of interest in terms of expanding the tool's functionality and needing a thorough study of these numerical methods and how they can be applied in the tool.

Finally, a pedagogical analysis of the suggested tool could be performed to investigate how the tool can be used to improve the teaching and learning of numerical methods and computer science, finding possible barriers or challenges to adoption and suggesting solutions to overcome them. These study and development avenues will not only enhance the suggested tool, but will also benefit the broader academic and professional communities in mathematics and computer science.